\section{Conclusion}
\label{sec:conclusion}
% LASER 构建了一个三步路线:BSE 保证线性计算严格等价,ASNC 将近似误差局限于唯一的非线性模块,而在这一高保真且结构一致的前向传播基础上,引入轻量的恒等梯度 STE 能够实现稳定可控的训练。实验结果表明,BSE 在编码上达到机器级精度,ASNC 的误差仅出现在单一可量化点上,STE 则在全局范围内保持稳定且无扩散。在 LLaMA-2 70B 上,MMLU 准确率仅下降 1.2%,而在 LLaMA-2 7B 上困惑度仅增加 +0.46。结合在 Loihi 2 上实现的约 200 倍能效提升,这些结果表明 LASER 在大规模脉冲模型中统一了高保真、可控性与可扩展性。
LASER\textendash SNN establishes a three\textendash step roadmap: BSE guarantees strict equivalence in linear computations, ASNC confines approximation error to a single nonlinear module, and—on this high\textendash fidelity, structurally consistent forward path—a lightweight identity\textendash gradient STE enables stable and controllable training. Experimental results demonstrate that BSE achieves machine\textendash level precision in encoding, ASNC introduces error only at one quantifiable point, and STE maintains stability without diffusion. On \mbox{LLaMA\textendash 2} 70B, accuracy on MMLU decreases by merely 1.2\%, while on \mbox{LLaMA\textendash 2} 7B, perplexity increases by only {+}0.46. Combined with a $200\times$ improvement in energy efficiency on Loihi~2, these results show that LASER\textendash SNN unifies fidelity, controllability, and scalability in large\textendash scale spiking models.